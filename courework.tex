\documentclass[a4paper,11pt]{article}
\usepackage{graphicx}
\usepackage{amsmath}
\usepackage{float}
\usepackage{caption}
\usepackage{geometry}
\geometry{a4paper, margin=1in}
\usepackage{fancyhdr}
\pagestyle{fancy}
\fancyhead{}
\fancyfoot{}
\fancyhead[L]{Coursework Title}
\fancyhead[R]{Your Full Name}
\fancyfoot[C]{\thepage}

\title{Theoretical and Computational Analysis of Stress Concentrations around Window Openings in Aircraft Fuselage}
\author{Your Full Name}
\date{}

\begin{document}
\maketitle

\begin{abstract}
This report presents an analysis of stress concentrations around window openings in an aircraft fuselage, examining theoretical approaches and computational simulations. The study assesses hoop and axial stresses and explores the impact of an elliptical window on the stress distribution.
\end{abstract}

\tableofcontents
\newpage

\section{Introduction}
\label{sec:introduction}
Introduce the objectives of the report and the importance of studying stress concentrations in aircraft fuselage structures. Discuss the layout of the report.

\section{Theory}
\label{sec:theory}
Provide an overview of relevant theoretical background. Include equations and derivations related to hoop stress, axial stress, and stress concentration factors. Proper referencing should be included here.

\section{Theoretical Analysis}
\label{sec:theoretical_analysis}
\subsection{Stress Analysis around a Window Opening}
Describe the setup: a thin-walled cylindrical fuselage with an elliptical window opening. Outline assumptions based on Appendix A, including fuselage diameter, wall thickness, cabin pressure, and window dimensions.

\subsection{Calculation of Hoop and Axial Stresses}
Present the derived equations and calculate the hoop and axial stresses for the fuselage structure. Include figures or tables where appropriate to show the results.

\subsection{Stress Distribution Analysis}
Explain the method for calculating tangential stress ($\sigma_{\beta\beta}$) around the rim of the window. Plot the distribution of tangential stress around the window opening and analyze the results.

\subsection{Material Considerations}
Discuss the yield stress of materials suitable for fuselage construction and relate them to the calculated maximum stress values. Assess assumptions and limitations.

\section{Computational Simulations}
\label{sec:computational_simulations}
\subsection{Simulation Setup}
Describe the computational model used to simulate stress concentrations. Explain the assumptions and parameters used in the simulation.

\subsection{Results}
Present the simulation results in either graphical or tabular form, as appropriate. Interpret the results and compare them to theoretical predictions.

\section{Discussion}
Critically assess the assumptions and methodologies applied in both theoretical and computational analyses. Discuss the implications of having multiple windows and potential effects on stress distribution.

\section{Conclusion}
Summarize key findings and highlight the significance of stress analysis in fuselage design. Provide insights or recommendations for future research.

\section{References}
Provide properly formatted references. Avoid referencing this coursework brief or course notes. Focus on reliable sources like journal articles, books, and other published works.

\appendix
\section{Appendix A: Data and Assumptions}
List any specific data or additional details used in the analysis, such as fuselage dimensions, material properties, and window specifications.

\end{document}
