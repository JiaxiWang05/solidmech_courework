\documentclass[a4paper,11pt]{article}
\usepackage{geometry}
\geometry{margin=1in}
\usepackage{graphicx}
\usepackage{amsmath}
\usepackage{hyperref}
\usepackage{float}
\usepackage{caption}
\usepackage{booktabs}
\usepackage{cleveref}
\hypersetup{
    colorlinks=true,
    linkcolor=blue,
    filecolor=magenta,
    urlcolor=cyan,
    pdftitle={Stress Concentrations in Solid Mechanics},
    pdfauthor={Your Full Name},
    bookmarksopen=true,
    bookmarksnumbered=true
}
\usepackage{titlesec}
\titleformat{\section}[block]{\bfseries\Large}{\thesection.}{1em}{}
\titleformat{\subsection}[block]{\bfseries\large}{\thesubsection.}{1em}{}
\titleformat{\subsubsection}[block]{\bfseries}{\thesubsubsection.}{1em}{}

\title{Stress Concentrations in Solid Mechanics}
\author{Your Full Name \\ ENGI 2221 - Solid Mechanics and Structures 2}
\date{January 13, 2025}

\begin{document}

\maketitle

\tableofcontents
\newpage

% Summary
\section*{Summary}
This report investigates the phenomenon of stress concentrations in engineering components. It includes a theoretical analysis of stress distributions around an elliptical window in a thin-walled cylinder and computational simulations of a plate with circular holes using Concept Analyst software. Key findings include the calculation of stress concentration factors (\(K_t\)) and their implications for material and structural design.

Use software like MATLAB, Python (Matplotlib), or Excel for graphs.
Use CAD or vector tools (e.g., AutoCAD, Inkscape) for custom figures.


\newpage

% Introduction
\section{Introduction}
\subsection{Background}
Stress concentrations are critical to engineering design, particularly in structural components like aircraft fuselages. These localized stress increases, often due to geometrical irregularities, can significantly influence a structure's durability.

\subsection{Objectives}
This report aims to:
\begin{itemize}
    \item Conduct a theoretical analysis of tangential stresses around an elliptical window in a thin-walled cylinder.
    \item Perform computational simulations to evaluate stress distributions and calculate stress concentration factors.
\end{itemize}

\subsection{Scope}
This report focuses on static loading conditions for a single elliptical window and circular holes, with key assumptions about uniform material properties and boundary conditions.

\newpage

% Theory
\section{Theory}
\subsection{Stress Concentration Fundamentals}
The stress concentration factor is defined as:
\[
K_t = \frac{\sigma_{\text{max}}}{\sigma_{\text{nom}}}
\]
For a circular hole in a large plate under uniaxial tension, \(K_t = 3\), while elliptical holes have \(K_t\) values dependent on their aspect ratio.

\subsection{Thin-Walled Cylinder Analysis}
In a pressurized thin-walled cylinder, the hoop stress is:
\[
\sigma_{\theta} = \frac{pD}{2t}
\]
where \(p\) is internal pressure, \(D\) is diameter, and \(t\) is wall thickness. The tangential stress \(\sigma_{\beta\beta}\) around the elliptical window is derived as:
\[
\sigma_{\beta\beta} = S \left(1 + \frac{2a}{b}\right)
\]

\subsection{Key Assumptions}
\begin{itemize}
    \item Uniform internal pressure.
    \item Single window with no nearby openings.
    \item Isotropic material behavior.
\end{itemize}

\newpage

% Theoretical Analysis
\section{Theoretical Analysis}
\subsection{Input Parameters}
The fuselage dimensions are:
\begin{itemize}
    \item Diameter: 5.96 m
    \item Wall thickness: 2 mm
    \item Cabin pressure: 75 kPa
    \item Window dimensions: \(a = 125 \, \text{mm}, b = 0.8a\)
\end{itemize}

\subsection{Tangential Stress Distribution}
The stress distribution around the elliptical window is shown in \Cref{fig:theoretical_stress}. The peak tangential stress occurs at the ends of the major axis.

\begin{figure}[H]
    \centering
    \includegraphics[width=0.8\textwidth]{figures/theoretical_stress.png}
    \caption{Tangential stress distribution around an elliptical window.}
    \label{fig:theoretical_stress}
\end{figure}

\subsection{Material Yield Strength Comparison}
The maximum stress is compared to typical fuselage materials, such as aluminum alloy with a yield strength of 300 MPa.

\newpage

% Computational Simulations
\section{Computational Analysis}
\subsection{Setup}
The plate geometry includes three circular holes arranged in an equilateral triangle with boundary conditions applied as shown in \Cref{fig:plate_geometry}.
\begin{figure}[H]
    \centering
    \includegraphics[width=0.7\textwidth]{figures/plate_geometry.png}
    \caption{Plate geometry with three circular holes.}
    \label{fig:plate_geometry}
\end{figure}

\subsection{Results}
\subsubsection{Variation of \(K_t\) with Geometry}
\Cref{fig:kt_vs_rd} shows how \(K_t\) varies with \(r/d\) and \(\theta\).

\begin{figure}[H]
    \centering
    \includegraphics[width=0.8\textwidth]{figures/kt_vs_rd.png}
    \caption{Stress concentration factor (\(K_t\)) vs. \(r/d\).}
    \label{fig:kt_vs_rd}
\end{figure}

\subsubsection{Effect of Plate Size}
The stress distribution changes as the plate size reduces, highlighting the influence of boundary effects.

\newpage

% Discussion
\section{Discussion}
\subsection{Comparison of Theoretical and Computational Results}
Theoretical predictions align well with computational results, with minor deviations attributed to boundary effects and assumptions in the analytical model.

\subsection{Engineering Implications}
Understanding stress concentrations is essential for designing robust fuselage structures, particularly in high-stress regions near window openings.

\subsection{Recommendations for Future Work}
Future studies should include:
\begin{itemize}
    \item Fatigue analysis under cyclic loading.
    \item Multi-window fuselage configurations.
\end{itemize}

\newpage

% Conclusion
\section{Conclusion}
This report successfully evaluated stress concentrations in both theoretical and computational frameworks. The findings emphasize the importance of considering stress concentrations in engineering design to prevent structural failures.

\newpage

% References
\section*{References}
\begin{enumerate}
    \item Book Title, Author, Year, Publisher.
    \item Journal Article, Author, Year, Journal Name, Volume, Pages.
\end{enumerate}

\newpage

% Appendices
\section*{Appendices}
\subsection*{Appendix A: Parameter Tables}
Include tables for fuselage and window dimensions.

\subsection*{Appendix B: Concept Analyst Setup}
Step-by-step guide for computational simulations.

\end{document}
